\begin{center}
    \textbf{Поля}
\end{center}
\begin{tabular}{l|l|l}
    $\displaystyle\operatorname{grad}{\phi} = \nabla\phi =  \frac{\partial{\phi}}{\partial{x}}\vec{i}+\frac{\partial{\phi}}{\partial{y}}\vec{j}+\frac{\partial{\phi}}{\partial{z}}\vec{k}$
    &
    $\displaystyle \operatorname{div}\vec{a} = (\nabla\cdot\vec{a}) = \cfrac{\partial a_x}{\partial x} + \cfrac{\partial a_y}{\partial y}+\cfrac{\partial a_z}{\partial z}$
    & 
    $\displaystyle\operatorname{rot}\vec{a} = (\nabla\times\vec{a})=
    \begin{vmatrix}
    \vec{i} & \vec{j} & \vec{k} \\
    \cfrac{\partial}{\partial x} & \cfrac{\partial}{\partial y} &
    \cfrac{\partial}{\partial z} \\
    a_x & a_y & a_z
    \end{vmatrix}$
    \\
    
    $\displaystyle\nabla{(\phi\cdot\vec{a})} = \phi\cdot\nabla\vec{a}+\nabla\phi\cdot\vec{a}$
    &
    $\displaystyle\operatorname{div}(\phi\cdot\vec{a}) = \phi\cdot\operatorname{div}\vec{a}+\operatorname{grad}{\phi}\cdot\vec{a}$
    & 
    $\displaystyle\operatorname{rot}(\phi\cdot\vec{a}) = \phi\operatorname{rot}\vec{a}+\operatorname{grad}\phi\times\vec{a}$
    \\
    
    $\displaystyle\nabla{f(u, v)} = \frac{\partial{f}}{\partial{u}}\nabla{u}+\frac{\partial{f}}{\partial{v}}\nabla{v}$
    &
    &
    $\displaystyle\nabla\times(\phi\vec{a}) = \phi(\nabla\times\vec{a})+\nabla\phi\times\vec{a}$
    \\
    
     $\displaystyle\operatorname{rot}(\operatorname{grad}\phi) = \vec{0}$
    &
    $\displaystyle\operatorname{div}(\vec{a}\times\vec{b}) = \vec{b}\operatorname{rot}\vec{a}-\vec{a}\operatorname{rot}\vec{b}$
    & 
    $\displaystyle\operatorname{rot}(\operatorname{rot}\vec{a}) = \operatorname{grad}(\operatorname{div}\vec{a}) - \Laplace\vec{a}$
    \\
    
    $\displaystyle\operatorname{div}(\operatorname{grad}\phi) = \Laplace\phi = \frac{\partial^2\phi}{\partial{x^2}} + \frac{\partial^2\phi}{\partial{y^2}} + \frac{\partial^2\phi}{\partial{z^2}}$
    &
     $\displaystyle\operatorname{div}(\operatorname{rot}\vec{a}) = 0$
    &
    \\
\end{tabular} 

\vspace{1ex}
Производная по направлению: $\displaystyle\frac{\partial\phi}{\partial{l}} = \frac{\partial\phi}{\partial{x}}\cos{\alpha} + \frac{\partial\phi}{\partial{y}}\cos{\beta} + \frac{\partial\phi}{\partial{z}}\cos{\gamma}$;
Векторные линии: 
$\displaystyle \frac{dx}{a_x} = \frac{dy}{a_y} = \frac{dz}{a_z}$
\\\\
\begin{tabular}{|l|}
	\hline
	\textbf{Нелинейные системы}: $\displaystyle\frac{dx}{P(x,y,z)} = \frac{dy}{Q(x,y,z)} = \frac{dz}{R(x,y,z)}$ \\
	I. \textit{Исключать} неизвестые к одному уравнению (получ. диффуру более высокого порядка) \\
	II. \textit{Интегрируемые комбинации}: свойство равных дробей: \\ 
	$\displaystyle \frac{a_1}{b_1} = \frac{a_2}{b_2} = ... = \frac{a_n}{b_n} = t$, то при $\forall k_1, k_2,...,k_n$ имеем
	$\displaystyle \frac{k_1a_1 + k_2a_2+...+k_na_n}{k_1b_1 + k_2b_2 +...+k_nb_n} = t$; \\ 
	\textit{Задача}: для $\displaystyle \frac{dx}{P},\frac{dy}{Q},\frac{dz}{R}$ подобрать $k_1,k_2$ и $k_3$, чтобы после комбинаций получить общий знаменатель \\
	(сокращается) и решить простую диффуру $\Rightarrow$ получим 1 кривую, выразим $\forall$ неизв. из нее \\ 
	и подставим в исходную систему $\Rightarrow$ получим диффуру из 2-ух оставшихся неизв.(ее решение - 2-ая кривая) \\
	\hline
\end{tabular}

