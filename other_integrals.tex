\begin{center}
    \textbf{Другие интегралы}
\end{center}

Формула Грина: $\displaystyle\oint\limits_{(K)}P(x, y)dx + Q(x, y)dy = \iint\limits_{(D)}\left(\frac{\partial Q}{\partial x} - \frac{\partial P}{\partial y}\right)dxdy$

Площадь: $S = \frac{1}{2}\oint\limits_{(K)}xdy - ydx$

Формула Стокса: $\displaystyle \oint\limits_{(K)} Pdx + Qdy + Rdz = \iint\limits_{(S)}\begin{vmatrix}
    \cos{\alpha} & \cos{\beta} & \cos{\gamma} \\
    \frac{\partial}{\partial{x}} & \frac{\partial}{\partial{y}} & \frac{\partial}{\partial{z}} \\
    P & Q & R \\
\end{vmatrix}dS =  \iint\limits_{(S)} \operatorname{rot}\vec{F} \vec{dS}$ 

$\displaystyle \frac{\partial{P}}{\partial{y}}=\frac{\partial{Q}}{\partial{x}},~ \frac{\partial{Q}}{\partial{z}}=\frac{\partial{R}}{\partial{y}},~ \frac{\partial{R}}{\partial{x}}=\frac{\partial{P}}{\partial{z}} \Rightarrow \int\limits_{\smile{AB}}Pdx+Qdy+Rdz = u(B) - u(A)$

Формула Гаусса-Остроградского: $\displaystyle \iiint\limits_{(V)} \operatorname{div}\vec{F}dxdydz = \iint\limits_{(S)} \vec{F}\vec{dS} = \iint\limits_{(S)}Pdxdy + Qdzdx + Rdxdy$ 

$\displaystyle V = \iiint\limits_{(T)}dxdydz = \frac{1}{3}\iint\limits_{(S)}xdydz+ydxdz+zdxdy = \frac{1}{3}\iint\limits_{(S)}(x\cos{\alpha} + y\cos{\beta} + z\cos{\gamma})dS$

\vspace{2ex}
\textit{Поверхностный интеграл}

$\displaystyle J = \iint\limits_{(S)}f(x, y, z)dS = \iint\limits_{(D)}f(x, y, z(x, y))\sqrt{1+(z_x')^2+(z_y')^2}dxdy = \iint\limits_{(\Delta)}f(x(u,v),y(u,v),z(u,v))\sqrt{EG - F^2}dudv$

$\displaystyle \vec{\tau_u} = (x_u',y_u',z_u'); ~\vec{\tau_v} = (x_v',y_v',z_v'); ~E = ||\vec{\tau_u}||^2; ~G = ||\vec{\tau_v}||^2; ~F = \vec{\tau_u} \cdot \vec{\tau_v}$
























