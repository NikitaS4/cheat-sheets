\begin{center}
	\textbf{Дифференциальные уравнения 1-ого порядка}
\end{center}

\begin{itemize}
	\item{\large Разрешенные относительно производной: $y' = f(x, y)$}
	
	\setlength{\extrarowheight}{1mm}
	\begin{tabular}{l}
		
		\begin{tabular}{|l|}
			\hline
			\textbf{С разделяющимися переменными:} \\
			$\displaystyle y' = f(x) g(y) $ \\
			$\displaystyle M(x)N(y)dx + P(x)Q(y)dy = 0$; \\
			\textbf{Решение:} $\displaystyle  \frac{-Q(y)dy}{N(y)} = \frac{M(x)dx}{P(x)}$ \\
			\hline
			Могут быть потеряны решения при делении \\
			\hline
		\end{tabular}
		\begin{tabular}{|l|}
			\hline
			\textbf{Однородные степени p:} \\
			ф-ия $u = f(x,y)$ наз. \textit{однородной степени p}, если \\ 
			$f(tx,ty) = t^p f(x,y)$; \\
			$\displaystyle M(x, y)dx + N(x, y)dy = 0$ - однородное ур-ние, если \\ 
			M и N однородные ф-ии одной степени; \\
			\textbf{Решение:} \textit{замена} $y = tx$, где $t = t(x)$ \\
			$dy = tdx + xdt$ \\	
			\hline
		\end{tabular}
		\\
		\\
		\begin{tabular}{|l|}
			\hline
			\textbf{Простейшие, привод. к однородному:} \\
			I. $\displaystyle \frac{dy}{dx} = 
			f\left( \frac{a_1x + b_1y + c_1}{a_2x + b_2y + c_2} \right), 
			c_1^2 + c_2^2 \neq 0$ \\
			\textbf{Решение:} \\
			1)$\begin{vmatrix}
				a_1 & b_1 \\
				a_2 & b_2 
			\end{vmatrix} \neq 0  
			\Rightarrow 
			\begin{cases} 
				x = \xi + \alpha \\
				y = \eta + \beta 
			\end{cases}$ \\
			$\displaystyle \frac{d\eta}{d\xi} = 
			f\left( \frac{a_1\xi + b_1\eta + a_1\alpha + b_1\beta + c_1}
						 {a_2\xi + b_2\eta + a_2\alpha + b_2\beta + c_2} 
		 	\right)$ \\
			$\begin{cases} 
				a_1\alpha + b_1\beta + c_1 = 0 \\
				a_2\alpha + b_2\beta + c_2 = 0 
			\end{cases} 
			\Rightarrow
			find~\alpha, \beta$ \\
			2)$\begin{vmatrix}
				a_1 & b_1 \\
				a_2 & b_2 
			\end{vmatrix} = 0  
			\Rightarrow$ \textit{замена:} $z = a_2x + b_2y$ \\\\
			II. замена $y = z^k$ ($k$ находят из условия \\
			однородности уравнения после подстановки) \\
			\hline		
		\end{tabular}
		\begin{tabular}{|l|}
			\hline
			\textbf{Линейные первого порядка:} \\
			I. \textit{Линейное однородное}: $y' + a(x)y = b(x)$ \\
			\textbf{Решение:} \\
			1) ищем в виде $y = u(x) \cdot v(x)$ \\
			2) решаем $u' + a(x)u = 0$, находим $u(x)$ (без +Сonst) \\
			3) решаем $v'\cdot u(x) = b(x)$, находим $v(x)$ (+Const) \\
			4) \textbf{NB}: при $\forall$ делении возвожна потеря решения! \\
			5) иногда возможно поменять $y$ и $x$: $x = u(y) \cdot v(y)$ \\\\
			
			II. \textit{Бернулли}: $y' + a(x)y = b(x)y^n\qquad(n \neq 1)$ \\
			\textbf{Решение:} \\
			1) делим ур-ние на $y^n$ \qquad 2) замена $\displaystyle \frac{1}{y^{n-1}}=z$ \\
			3) свели к задаче выше \\\\
			
			III. \textit{Риккати}: $y' + a(x)y + b(x)y^2=c(x)$ \\
			\textbf{Решение:} \\
			1) замена: $y = \tilde{y}(x) + z$, где $\tilde{y}(x)$ - частное решение \\
			2)для $\displaystyle y' + 2y^2=\frac{6}{x^2}$ ищем что-то	в виде $\displaystyle y = \frac{a}{x}$ \\
			3)для $y' = y^2 = x^2 - 2x$ ищем что-то в виде $y=ax + b$ \\	
			\hline	
		\end{tabular}
		\\
		\begin{tabular}{|l|}
			\hline
			\textbf{Уравнение в полных дифференциалах:} \\
			$\displaystyle M(x, y)dx + N(x, y)dy = 0$ наз. \\ 
			\textit{в полных дифференциалах}, если \\
			$\displaystyle\frac{\partial M}{\partial y} \equiv \frac{\partial N}{\partial x}$ \\
			\textbf{Решение:} \\
			1) Вид общего решения: $F(x,y) = C$ \\
			2) $F = \int M(x,y)dx = M_{primitive}(x,y) + \phi(y)$ \\
			3) $( M_{primitive}(x,y) + \phi(y) )_{y}' = N(x,y)$; \\ 
			находим $\phi(y)$ \\
			4) ответ: $M_{primitive}(x,y) + \phi(y) = C$ \\
			\hline
			\hline
			\textbf{Дарбу:} \\
			$\displaystyle M(x, y)dx + N(x, y)dy + P(x,y)(xdy - ydx) = 0$ \\
			где $M(x,y)$ и $N(x,y)$ - однород. ф-ии степ. $m$, \\
			а $P(x,y)$ - однород. ф-ия степ. $l (\neq m - 1)$ \\
			\textbf{Решение:} \\
			\textit{замена} $y = zx$, получим Бернулли \\
			\hline
		\end{tabular}
		\begin{tabular}{|l|}
			\hline
			\textbf{Интегрирующий множитель $\mu$}: \\
			ф-ия $\mu(x,y) \neq 0$, после умножения на кот. ур-ние \\
			$M(x, y)dx + N(x, y)dy = 0$ превращается в ур-ние \\
			\textit{в полных диф-лах}. \\
			\textbf{Решение:} \\
			1) Возможно, он легко в уме подберется(нет) \\
			2) пусть $\omega = \omega(x,y)$ и $ \mu = \mu(\omega)$; тогда:  \\
			$\displaystyle \frac{\mu(\omega)'}{\mu(\omega)} = 
			\frac{\frac{\partial N}{\partial x} - 
				  \frac{\partial M}{\partial y}}
			{\frac{\partial \omega}{\partial y}M(x,y) - 
			 \frac{\partial \omega}{\partial x}N(x,y)}$ \\
		 3) цель: \textit{подобрать} такую ф-ию $\omega(x, y)$, чтобы \\
		 выражение справа представилось некой ф-ией $g(\omega)$; \\ 
		 тогда сможем найти $\mu(\omega) = \mu(x,y)$, решив диффуру \\
		 4) возможные варианты $\omega(x,y)$: \\ 
		 $\omega = \pm x$ \qquad $\omega = \pm y$ \qquad $\omega = \pm xy$ \\ 
		 $\omega = x \pm y$ \qquad $\displaystyle \omega = \pm \frac{x}{y}$ \qquad $\displaystyle \omega = \pm \frac{y}{x}$ \\
			
			\hline
		\end{tabular}
			
	\end{tabular}

%\newpage

%	\item {\large Не разрешенные относительно производной: $F(x,y,y') = 0$}
%
%	todo

\end{itemize}
\setlength{\extrarowheight}{0mm}